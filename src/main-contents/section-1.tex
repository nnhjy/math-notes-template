\section{Section 1} \label{section-1}

The first paragraph is not indented by default. Use \texttt{\textbackslash usepackage\{indentfirst\}} to intent first paragraphs.

In ordinary text, we can write $\mathbi{a} \in \R$, compared with $\mathbf{a} \in \R$.

\definition \cite{einstein} is an in-text citation. 

\concept Concept is supplementary to definitions

Text following theorem denotation is italic by default.

\rm % Remove the above formatting (e.g. italics) for the following text.

Use \texttt{\textbackslash rm} to obtain ordinary texts with default indent.

\noindent Ordinary text without indent. \href{https://www.overleaf.com/learn/latex/Hyperlinks}{Hyperlink} is available. 

Referring to Section \ref{section-2} and equation \ref{eq:2.1}. More details can be found \href{https://www.overleaf.com/learn/latex/Cross_referencing_sections%2C_equations_and_floats}{here}.

\thm Let $K$ be a compact set in a metric space $(X,d)$. Suppose $\mathcal{F}=\{U_\alpha\}_{\alpha \in A}$ is an open cover of $K$, then there exists a positive number $\lambda$ so that for every $p \in K$ the open ball $B(p,\lambda)$ is contained in one of the open sets of $\mathcal{F}$.

\begin{proof}

Since $K \subset \underset{\alpha \in A}\cup U_\alpha$, for each point $p$ in $K$ there is a positive number $2\varepsilon(p)$ so that the ball $B(p,2\varepsilon(p))$ is contained in one of the open sets of $\mathcal{F}$. Clearly $\{B(p,2\varepsilon(p)\}_{p \in K}$ forms an open cover of K, and so by compactness this admits a finite refinement.

\end{proof}

\rm % Remove the above formatting (e.g. italics) for the following text.